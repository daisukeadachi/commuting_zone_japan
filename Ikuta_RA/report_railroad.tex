\documentclass{ltjsarticle}
% \usepackage{url}
\usepackage{graphicx}
\title{CZ・UEAと陸上交通網}
\author{生田直輝}
\begin{document}
\maketitle

\begin{itemize}
  \item CZ・UEAと陸上交通網の関連を可視化するため、それぞれのコロプレス図に鉄道・高速道路を重ね合わせた。作成した各図については以下の通り
  \begin{itemize}
    \item Original(市町村境界を\underbar{基準化していない}もの。)
    \begin{itemize}
      \item 鉄道(Railroad)
      \begin{description}
        \item[\ref{allCZandRail}] データあるすべての年のCZと鉄道の地図を並べたもの
        \item[\ref{KanCZandRail}] データあるすべての年のCZと鉄道の地図を並べたもの
        \item[\ref{allUEAandRail}] データあるすべての年のUEAと鉄道の地図を並べたもの
        \item[] 
      \end{description}
      \item 高速道路(Expway)
    \end{itemize}
  \end{itemize}
\end{itemize}


\begin{figure}[pbth]
  \centering
  \includegraphics[width=150mm]{output/map_image/Railroad/Original/multiple/1980to2015_CZmap.png}
  \caption{\label{allCZandRail}全国のCZと鉄道}
\end{figure}

\begin{figure}[pbth]
  \centering
  \includegraphics[width=150mm]{output/map_image/Railroad/Original/multiple/1980to2015_UEAmap.png}
  \caption{\label{allUEAandRail}全国のUEAと鉄道}
\end{figure}

\begin{figure}[pbth]
  \centering
  \includegraphics[width=150mm]{output/map_image/Railroad/Original/multiple/1980to2015_CZmap_kanto.png}
  \caption{\label{KanCZandRail}関東のCZと鉄道}
\end{figure}

\begin{figure}[pbth]
  \centering
  \includegraphics[width=150mm]{output/map_image/Railroad/Original/multiple/1980to2015_UEAmap_kanto.png}
  \caption{\label{KanUEAandRail}関東のUEAと鉄道}
\end{figure}



\end{document}
